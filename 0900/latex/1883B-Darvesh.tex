\documentclass{article}
\usepackage{graphicx} % Required for inserting images
\usepackage{hyperref}
\usepackage{listings}
\usepackage{color}

% Example : https://codeforces.com/problemset/problem/266/A 
% So the title would be 1883B - Chemistry
\title{1883B - Chemistry} 

% Author must be your full name
\author{Darvesh Aziz Mawla} 

% Date is when you create this report
\date{23 April 2024}

\begin{document}

\maketitle

% There are 4 Sections, Problem, Objective, Solution, Code

% Problem section contains hyperlink to the problem
\section{Problem}

Problem Description : \href{https://codeforces.com/problemset/problem/1883/B}{https://codeforces.com/problemset/problem/1883/B}

% Objective section contains what is the problem's objective
\section{Objective}

We need to check if its possible to remove exactly k characters from the string given and check if its possible to rearrange the remaining characters into a palindrome

A palindrome is a string that reads the same forwards and backwards. For example, the strings "z", "aaa", "aba", "abccba" are palindromes, while the strings "codeforces", "reality", "ab" are not.

example :
input :

4

1 0

a

2 0

ab

2 1

ba

3 1

abb

explanation :
In the first test case, nothing can be removed, and the string "a" is a palindrome.

In the second test case, nothing can be removed, but the strings "ab" and "ba" are not palindromes.

In the third test case, any character can be removed, and the resulting string will be a palindrome.

In the fourth test case, one occurrence of the character "a" can be removed, resulting in the string "bb", which is a palindrome.


% Solution section contains how you approch the problem and your solution
\section{Solution}

We need to make sure there aren't too many letters appearing an odd number of times (let's call this x). If x is greater than k+1, we can't solve the problem because we can't fix it with k moves.

But if x is k+1 or less, we can make it work. We just keep removing letters that appear an odd number of times until either there's only one left or none at all.

So, we just need to check if x is below or equal to k+1 to check if each testcase can be solved or no.

% Code section contains your solution code

\newpage
\section{Code}

\lstset{language=C++,
        basicstyle=\ttfamily,
        keywordstyle=\color{blue}\ttfamily,
        stringstyle=\color{red}\ttfamily,
        commentstyle=\color{green}\ttfamily,
        morecomment=[l][\color{magenta}]{\#}
}
\begin{lstlisting}

#include<bits/stdc++.h>
using namespace std;

int main(){
    int t;
    cin >> t;
    while(t-- > 0){
        int freq[26] = {0};
        string s;
        int n,k;
        cin >> n >> k >> s;
        for (auto &ch : s) { 
            freq[(int)static_cast<unsigned char>( ch ) - 97]++;
        } 
        int oddsum = 0;
        for(int i=0;i<26;i++){
            if(freq[i] % 2 != 0){
                oddsum++;
            }   
        }
        if(oddsum <= k +1){
            cout << "YES" << endl;
        }
        else{
            cout << "NO" << endl;
        }
    }
}

\end{lstlisting}

\end{document}
