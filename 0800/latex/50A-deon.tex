\documentclass{article}
\usepackage{graphicx} % Required for inserting images
\usepackage{hyperref}
\usepackage{listings}
\usepackage{color}

% Example : https://codeforces.com/problemset/problem/266/A 
% So the title would be 266A - Stones on the Table
\title{50A - Domino piling} 

% Author must be your full name
\author{Deon Marshal} 

% Date is when you create this report
\date{14 April 2024}

\begin{document}

\maketitle

% There are 4 Sections, Problem, Objective, Solution, Code

% Problem section contains hyperlink to the problem
\section{Problem}

Problem Description : \href{https://codeforces.com/problemset/problem/50/A}

% Objective section contains what is the problem's objective
\section{Objective}

To find out how many dominoes with a size of 2 x 1 can be used to cover a box measuring m x n, where the domino part must not cross the edge of the box
Example : 

m & n are 2 & 4
then the maximum number of dominoes needed is 4 to cover the surface of the box even though there is 1 x 1 more box that is not covered

% Solution section contains how you approch the problem and your solution
\section{Solution}

By allowing dominoes to be rotated vertically and horizontally, the placement of dominoes with a size of 2 x 1 will not be limited by mismatched positions. So, we can calculate it by dividing the area of the domino by rounding down.
% Code section contains your solution code

\newpage
\section{Code}

\lstset{language=C++,
        basicstyle=\ttfamily,
        keywordstyle=\color{blue}\ttfamily,
        stringstyle=\color{red}\ttfamily,
        commentstyle=\color{green}\ttfamily,
        morecomment=[l][\color{magenta}]{\#}
}
\begin{lstlisting}

#include <bits/stdc++.h>
#include <iostream>
#include <vector>
#include <algorithm>
#include <cmath>
using namespace std;

int main() {
     int n,k;
     cin>>n>>k;
     cout<<(n*k)/2;
    return 0;
}

\end{lstlisting}

\end{document}
