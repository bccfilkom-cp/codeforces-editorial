\documentclass{article}
\usepackage{graphicx} % Required for inserting images
\usepackage{hyperref}
\usepackage{listings}
\usepackage{color}

% Example : https://codeforces.com/problemset/problem/266/A 
% So the title would be 266A - Stones on the Table
\title{1899B - 250 Thousand Tons of TNT} 

% Author must be your full name
\author{Deon Marshal} 

% Date is when you create this report
\date{24 April 2024}

\begin{document}

\maketitle

% There are 4 Sections, Problem, Objective, Solution, Code

% Problem section contains hyperlink to the problem
\section{Problem}

Problem Description : \href{https://codeforces.com/problemset/problem/1899/B}

% Objective section contains what is the problem's objective
\section{Objective}

Look for the largest difference in the mass of TNT boxes carried by each truck in line
The placement of each box has a different mass, but each truck must carry the same number of boxes. 
The program is asked to analyze all possibilities and look for events where two trucks have a maximum mass difference with the appropriate event conditions.
Example : 

6 boxes with a mass of 10 2 3 6 1 3, when transported on a truck the biggest difference between trucks is 9 where each truck lifts 1 box and the first truck will have a difference of 9 with the fifth truck

% Solution section contains how you approch the problem and your solution
\section{Solution}

The solution uses a brute force technique where the program will look for differences in each iteration. 
The first iteration is to ensure that each truck will carry the same number of boxes, which means the modulo value is 0 in each iteration. 
Then the second repetition looks for the difference between each box and the box after it which will be stored in one vector. 
To make calculations easier, a vector containing the cumulative number of boxes is previously prepared so that when searching for the difference, 
the program does not take too long to count the boxes one by one again.
For the largest difference result, simply sort the difference vector and then call the first value of the vector. 
This value will be compared again in the next iteration
% Code section contains your solution code

\newpage
\section{Code}

\lstset{language=C++,
        basicstyle=\ttfamily,
        keywordstyle=\color{blue}\ttfamily,
        stringstyle=\color{red}\ttfamily,
        commentstyle=\color{green}\ttfamily,
        morecomment=[l][\color{magenta}]{\#}
}
\begin{lstlisting}

#include <bits/stdc++.h>
#include <iostream>
#include <vector>
#include <algorithm>
using namespace std;

int main() {
    int t;
    cin >> t;
    while (t--) {
        int n, resMax = 0;
        cin >> n;
        vector<int> kumulatif;
        int sum = 0;
        kumulatif.push_back(0);
        for (int i = 0; i < n; i++) {
            int b;
            cin >> b;
            if(b==0) b = 1;
            sum += b;
            kumulatif.push_back(sum);
        }

        for (int i = 1; i <= n; ++i) {
            vector<int> selisih;
            if (n % i == 0) {
                for (int j = i; j <= n; j += i) {
                    int sum = kumulatif[j] - kumulatif[j - i];
                    selisih.push_back(sum);
                }
                sort(selisih.begin(), selisih.end());
                resMax = max(resMax, selisih[selisih.size() - 1] - selisih[0]);
            }
        }

        cout << resMax << endl;
    }
    return 0;
}

\end{lstlisting}

\end{document}
