\documentclass{article}
\usepackage{graphicx} % Required for inserting images
\usepackage{hyperref}
\usepackage{listings}
\usepackage{color}

% Example : https://codeforces.com/problemset/problem/266/A 
% So the title would be 266A - Stones on the Table
\title{50A - Domino piling} 

% Author must be your full name
\author{Muhammad Arif Rifki} 

% Date is when you create this report
\date{1 May 2024}

\begin{document}

\maketitle

% There are 4 Sections, Problem, Objective, Solution, Code

% Problem section contains hyperlink to the problem
\section{Problem}

Problem Description : \href{https://codeforces.com/problemset/problem/50/A}{https://codeforces.com/problemset/problem/50/A}

% Objective section contains what is the problem's objective
\section{Objective}
You are given a rectangular board of M × N squares. Also you are given an unlimited number of standard domino pieces of 2 × 1 squares. You are allowed to rotate the pieces. You are asked to place as many dominoes as possible on the board so as to meet the following conditions:


1. Each domino completely covers two squares.

2. No two dominoes overlap.

3. Each domino lies entirely inside the board. It is allowed to touch the edges of the board.

Find the maximum number of dominoes, which can be placed under these restrictions.
\\\textbf{Input}

In a single line you are given two integers M and N — board sizes in squares \(1 \leq M \leq N \leq 16\).


\\\textbf{Output}

Output one number — the maximal number of dominoes, which can be placed.

% Solution section contains how you approch the problem and your solution
\section{Solution}

The solution to this problem is, you have to find a multiple of 2 that is close to and less than or equal to $m x n$

% Code section contains your solution code

\newpage
\section{Code}

\lstset{language=C++,
        basicstyle=\ttfamily,
        keywordstyle=\color{blue}\ttfamily,
        stringstyle=\color{red}\ttfamily,
        commentstyle=\color{green}\ttfamily,
        morecomment=[l][\color{magenta}]{\#}
}
\begin{lstlisting}
#include <bits/stdc++.h>
using namespace std;
int main()
{
    int m, n, cnt = 0, temp = 0;
    cin >> m >> n;
    while (m * n >= temp + 2)
    {
        temp += 2;
        cnt++;
    }
    cout << cnt;
    return 0;
}

\end{lstlisting}

\end{document}